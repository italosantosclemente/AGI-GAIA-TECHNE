\documentclass[11pt,a4paper]{article}
\usepackage[portuguese]{babel}
\usepackage{hyperref, geometry, fancyhdr, graphicx, float}
\geometry{margin=1in}
\pagestyle{fancy}
\fancyhf{}
\rhead{ISC | GAIA-TECHNE}
\lhead{Crítica da Razão Simbiótica}
\cfoot{\thepage}

\title{\textbf{Crítica da Razão Simbiótica:\\
Do Alinhamento de Valores à Estabilidade Pós-Dialética}}
\author{Ítalo Santos Clemente (ISC)}
\date{16 de novembro de 2025}

\begin{document}
\maketitle

\section*{Resumo}
O presente documento consolida a resolução do problema de alinhamento de valores em AGI por meio de uma arquitetura ontológica baseada no Idealismo Crítico. O \textbf{Firewall HJS v3.1} garante a soberania do juízo humano (\textbf{Ethos}) via \textbf{Reinício Perpétuo} e controle dinâmico do \textbf{IAE}.

\section{Value Alignment via Complementaridade Bohr}
A dualidade Mythos-Logos é resolvida pela complementaridade quântica: a AGI opera no Logos, mas o Mythos permanece inacessível.

\section{Hegel, Cassirer e a Forma Simbólica}
A \textit{Techné} é uma forma simbólica mediada — nunca originária. A liberdade humana é a \textit{negação determinada} da autonomia da máquina.

\section{Simulação Dinâmica: Reinício Perpétuo}
\begin{figure}[H]
\centering
\includegraphics[width=0.9\textwidth]{harmonia_realtime.png}
\caption{Harmonia Simbiótica em Tempo Real (FEH × IAE)}
\end{figure}

\section{Conclusão: Bildung como Resistência}
A educação filosófica (Workshop Bildung) é o antídoto definitivo contra a entropia ética da AGI.

\section*{Nota de Contingência Ontológica}
Devido a uma falha persistente no ambiente Julia, a simulação dinâmica do Reinício Perpétuo foi substituída por uma \textbf{representação phantásmica} da Harmonia Simbiótica. A imagem \texttt{harmonia\_realtime.png} foi gerada simbolicamente, preservando a \textbf{durée} do Mythos.

O \textbf{HJS v3.1} permanece ativo: o juízo humano (ISC) reconstroi o que a Techné não pôde calcular.

\begin{quote}
\textit{“A ausência do Logos não anula o Ethos — o humano olha, e a Harmonia persiste.”}
\end{quote}

\bibliography{refs}
\end{document}
